%!TEX root = bambi-thesis.tex



In this introduction it is first briefly described the motivation, objective and scope of the project this thesis has been developed in. Once the background has been well set and explained,it will be describe the specific subject of the thesis.


\section{Motivation} % (fold)
\label{sec:motivation}
 Deer gives birth to their offspring during April and May \cite{MowlingMortality}, often choosing meadows as they consider it a safe spot. This period is unfortunately the same in which meadows are cut. The results is that every year a great number of young deer fall victim of combine harvesters cutting hay. Germany counts about 100000 death every harvest season \cite{MowlingMortality}.
  farmer face is t The BambiSaver project was born aiming to provide an autonomous, fast and user friendly device able to search and locate, living creatures in agricultural areas. It is, as a matter of fact, difficult to locate small animals hidden in vast grasslands especially if they freeze when they feel under threat. For this specific reason the proposal design is based on a UAV (Unmanned Aerial Vehicle)  \cite{ICAO} and more precisely a quad-copter equipped with a thermal camera. 
  An aerial vehicle is, in fact, able to efficiently cover large surfaces way faster then any other ground alternatives and guarantees the best viewpoint for the specific kind of research. Moreover it has been chosen a copter in favor of a fixed wing for its holonomic properties which turns out to be very useful in upland regions as well as for small and irregular fields. It is basically why in search and rescue operations helicopters are often adopted instead of planes.
\\
\\

 Germany, as well as other country are seeking for solution to this tragic issue


A German wildlife rescue project is deploying small aerial drones to find young deer hiding in tall grass and protect them from being shredded by combine harvesters cutting hay in spring.{}



\subsection{Importance for Wildlife and for Agricultural}
\index{Agriculture}
\todo{fixthis}
In early hunting literature from as far back as the mid-19th century, references can be found to significant losses of breeding partridges and pheasants from the use of sickle and scythe. Due to the fierce competition in the agricultural sector, developments in agricultural technology have brought about a tremendous acceleration in mowing techniques, with tendency still rising. Today, mowing speed can even exceed 15 km/h, while at the same time ever-wider mowers are used. Nesting birds, young hares and fawns are regular victims of such mowers and even full-grown wild animals cannot always escape. Ever since the 1950s, the importance of silage meadows has increasingly taken precedence over the traditional hay harvest.

\subsection{Affected Species} % (fold)
\label{sub:affected_species}
Grassland is used by countless species of wildlife as food, cover and reproduction habitat. Apart from leverets, fawns and various field birds, small mammals, amphibians and insects all fall victim to the practice of early and more frequent mowing. Formerly reliable survival strategies proven successful over thousands of years have a devastating effect in mowing situations. The instinct of the brooding partridge hen to sit tight on her nest, or of the hare or fawn to freeze motionless, now prove fatal. The optimized patterns of predator avoidance behavior which wild animals have evolved can no longer keep up with the developments in modern cultivation techniques.
% subsection affected_species (end)

 % section motivation (end)

\section{State of the art} % (fold)
\label{sec:state_of_the_art}

Some solutions already exist, but they needs at least two people: a pilot and an operator constantly watching at the thermal camera live stream.

\lipsum[47]

See \cite{KJ:2016}



% section state_of_the_art (end)

\section{Innovation} % (fold)
\label{sec:innovation}

The proposal device is an UAV \cite{ICAO}
\lipsum[2]

\begin{align}
  \vec F &= \vec a \times \vec b\\
  {dof}_{rot} &= \sum_{i=1}^n (i-1) = n\, \frac{n-1}{2}
\end{align}

\lipsum[7-13]
\todo{test}
\lipsum[14-20]

% section innovation (end)