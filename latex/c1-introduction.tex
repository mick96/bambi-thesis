%!TEX root = bambi-thesis.tex



In this introduction it is first briefly described the motivation, objective and scope of the project this thesis has been developed. Then, the Thesis outline for every Chapter is described.


\section{Motivation} % (fold)
\label{sec:motivation}
 Deer gives birth to their offspring in spring, especially during April and May \cite{MowlingMortality}, often choosing meadows as they consider it a safe spot. This period is unfortunately the same in which meadows are cut. The results is that every year a great number of young deer fall victim of combine harvesters cutting hay. Germany counts about 100000 death every harvest season \cite{MowlingMortality}. The BambiSaver project was born with the aim to provide an autonomous, fast and user friendly device able to detect and localize living creatures in agricultural area.




Parlare un po dell'idea di base 




 Germany, as well as other country are seeking for solution to this tragic localize animal


A German wildlife rescue project is deploying small aerial drones to find young deer hiding in tall grass and protect them from being shredded by combine harvesters cutting hay in spring.{}

The proposal thesis was designed 

\subsection{Importance in agricultural}
\lipsum[7]
\index{Agriculture}
\todo{fixthis}

% section motivation (end)

\section{State of the art} % (fold)
\label{sec:state_of_the_art}

Some solutions already exist, but they needs at least two people: a pilot and an operator constantly watching at the thermal camera live stream.

\lipsum[47]

See \cite{KJ:2016}



% section state_of_the_art (end)

\section{Innovation} % (fold)
\label{sec:innovation}

\lipsum[2]

\begin{align}
  \vec F &= \vec a \times \vec b\\
  {dof}_{rot} &= \sum_{i=1}^n (i-1) = n\, \frac{n-1}{2}
\end{align}

\lipsum[7-13]
\todo{test}
\lipsum[14-20]

% section innovation (end)