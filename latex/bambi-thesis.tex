\RequirePackage[l2tabu, orthodox]{nag}
\documentclass[a4paper,12pt, openright, twoside]{report}
\usepackage{listings}
\usepackage{color} %red, green, blue, yellow, cyan, magenta, black, white
\usepackage[T1]{fontenc}
\usepackage[utf8]{inputenc}
\usepackage[american]{babel}
\usepackage[breaklinks=true,colorlinks=false,pdfborder={0 0 0}]{hyperref}
\usepackage{lmodern}
\usepackage{graphicx}
\usepackage{datetime2}
\usepackage{url}
%\usepackage{breakurl}
\usepackage{wrapfig}
\usepackage{subcaption}
\usepackage{microtype}
%\usepackage{inconsolata}
%\usepackage[lf]{MinionPro}
\usepackage{textcomp}
\usepackage{booktabs}
\usepackage{multirow}
\usepackage{amsmath}
\usepackage{amssymb}
\usepackage{cleveref}
\usepackage{courier}
\usepackage{xstring}
\usepackage{mathrsfs}
\usepackage{units}
\usepackage{threeparttable}
\usepackage[inline]{enumitem}
\usepackage{todonotes}
\usepackage{gensymb}
\usepackage{lipsum}
\usepackage{fancyhdr}
\usepackage[headsep=.5cm,headheight=1cm,left=3cm,right=2cm,top=3cm,bottom=2.5cm]{geometry}
\usepackage[numbib]{tocbibind}
\usepackage{appendix}
\usepackage{imakeidx}
\usepackage{tabularx}
\usepackage[ruled,vlined]{algorithm2e}
\usepackage[toc,acronym]{glossaries}
\usepackage{gensymb}
\usepackage{float}
\usepackage{pdfpages} 
\floatstyle{plain}
\newfloat{Code}{H}{myc}

% Set listing style
\lstdefinestyle{mystyle}{
captionpos=b,
frame=tb,
prebreak=\raisebox{0ex}[0ex][0ex]{\ensuremath{\hookleftarrow}},
aboveskip=20pt,
belowskip=20pt
}
\lstset{style=mystyle}

\definecolor{maroon}{rgb}{0.5,0,0}
\definecolor{darkgreen}{rgb}{0,0.5,0}
\lstdefinelanguage{XML}
{
  basicstyle=\ttfamily,
  morestring=[s]{"}{"},
  morecomment=[s]{?}{?},
  morecomment=[s]{!--}{--},
  commentstyle=\color{darkgreen},
  moredelim=[s][\color{black}]{>}{<},
  moredelim=[s][\color{red}]{\ }{=},
  stringstyle=\color{blue},
  identifierstyle=\color{maroon}
}

\makeindex
\indexsetup{level=\section,noclearpage}

\reversemarginpar
\setlength{\marginparwidth}{2cm}

\providecommand{\keywords}[1]{\textbf{\textit{Keywords: }} #1}
\def\UrlBreaks{\do\/\do-}
\input{matlab-code-config}


\title{Agricultural Field Detection and Coverage Path Planning for an Unmanned Aerial Vehicle}
\author{Michael Rimondi}
\date{\today}


% Make the list of * output as SECTION rather than CHAPTER
\makeatletter
\newcommand\renewlistof[3]%
   {\renewcommand#1%
      {\section{#3}%
       %\addcontentsline{toc}{chapter}{#3}%
       \markboth{#3}{#3}%
       \@starttoc{#2}%
      }%
   }

\makeatother
\renewlistof\listoffigures{lof}{\listfigurename}
\renewlistof\listoftables{lot}{\listtablename}
\renewlistof\lstlistoflistings{lstlol}{\lstlistlistingname}

\raggedbottom

%!TEX root = bambi-thesis.tex

\def\maketitle{\begingroup % Create the command for including the title page in the document
% \centering % Center all text
% \thispagestyle{empty} 
% \vspace*{4cm}

% \rule{\textwidth}{1.6pt}\vspace*{-\baselineskip}\vspace*{2pt} % Thick horizontal line
% \rule{\textwidth}{0.4pt}\\[\baselineskip] % Thin horizontal line

% {\LARGE  \textsc{\@title}}\\[0.2\baselineskip] % Title

% \rule{\textwidth}{0.4pt}\vspace*{-\baselineskip}\vspace{3.2pt} % Thin horizontal line
% \rule{\textwidth}{1.6pt}\\[\baselineskip] % Thick horizontal line

% \scshape % Small caps
% \subtitle\\[1.5\baselineskip] % Tagline(s) or further description
% Bologna, \@date \par % Location and year

% \vspace*{2\baselineskip} % Whitespace between location/year and editors

% \subject

% \vspace*{2\baselineskip} % Whitespace between location/year and editors

% Edited by \\[\baselineskip]
% {\Large \@author\\[2\baselineskip]\par} % Editor list
% \includegraphics[width=.15\textwidth]{figures/universitiy-of-bologna-logo.jpg}\\[0.3\baselineskip]
% {\itshape Università di Bologna\par} % Editor affiliation
% \vfill % Whitespace between editor names and publisher logo



% \vspace*{\baselineskip} % Whitespace between location/year and editors

% \textit{\textbf{Supervisors:}\\[\baselineskip]\textsc{\mentors}}

\thispagestyle{empty} 
\begin{center}
{{\Large{\textsc{Alma Mater Studiorum $\cdot$ Università di
Bologna}}}}\\
\vskip - 5pt
\rule{\textwidth}{0.1mm}
\rule[0.5cm]{\textwidth}{0.6mm}
\vskip -13pt
{\small{\textsc{Scuola di Ingegneria e Architettura\\
\vspace{4mm}
\small{Dipartimento di Ingegneria dell'Energia Elettrica e dell'Informazione \\
"Guglielmo Marconi" – DEI\\}}}}
% 
% 
% SCUOLA DI INGEGNERIA E ARCHITETTURA
% DIPARTIMENTO DI
% INGEGNERIA DELL'ENERGIA ELETTRICA E DELL'INFORMAZIONE
% "Guglielmo Marconi"
% DEI
% 
\vspace{15mm}
{\large{\textsc{{Corso di Laurea in Ingegneria dell'Automazione}}}}
\vskip 5mm
\rule{10cm}{0.4mm}
\rule[0.5cm]{10cm}{0.1mm}\\
\vskip -5pt
{\LARGE\textbf{{Agricultural Field Detection}}}\\
\vspace{.5em}
{\LARGE{\textbf{{and Path Planning}}}}\\
\vspace{.5em}
{\LARGE{\textbf{{for an Unmanned Aerial Vehicle}}}}\\
%\vspace{3mm}
%{\LARGE{\textbf{ --}}}\\
\vskip -2pt
\rule{10cm}{0.1mm}
\rule[0.5cm]{10cm}{0.4mm}\\
\vspace{5mm} {\large{\textsc{Progetto di Laurea}}}
\end{center}
\vspace{8mm}
\centering
\includegraphics[width=.25\textwidth]{figures/university-of-bologna.pdf}
\vfill
\par
\noindent
\begin{minipage}[t]{0.47\textwidth}
{\large{\textit{Relatore:}}\\
{\textbf{Chiar.mo Prof. Lorenzo Marconi}}}\\
\vskip 8pt
{\large{\textit{Correlatore:}}\\
{\textbf{Prof. Nicola Mimmo}}}
\end{minipage}
\hfill
\begin{minipage}[t]{0.47\textwidth}\raggedleft
{\large{\textit{Presentato da:}}\\
%\vspace{2mm}
{\textbf{Florian Mahlknecht}}}
\end{minipage}
\vspace{22mm}
\begin{center}
{\large{\textsc{II Sessione\\%inserire il numero della sessione in cui ci si laurea
Anno Accademico 2017/2018}}}%inserire l'anno accademico a cui si è iscritti
\end{center}



\newpage

\endgroup}
\makeatother


% clear fancy hdr spaces
\fancyhf{}
\renewcommand*{\sectionmark}[1]{ \markright{\thesection\ ##1} }
% \fancyhead[LE,RO]{\rightmark}
\fancyhead[LO,RE]{\leftmark}
\fancyfoot[C]{\thepage}
\pagestyle{fancy}

%%%%%%%%%%%%%%%%%%%%%%%%%%Acronyms list BEGIN%%%%%%%%%%%%%%%%%%%%%%%%%%
\makeglossaries
\newacronym{uav}{UAV}{Unmanned Aerial Vehicle}
\newacronym{ros}{ROS}{Robot Operating System}
\newacronym{cpp}{CPP}{Coverage Path Planning}
\newacronym{wgs84}{WGS84}{World Geodetic System of 1984}
\newacronym{utm}{UTM}{Universal Transverse of Mercator}
\newacronym{sitl}{SITL}{Software-In-The-Loop}
\newacronym{qgc}{QGC}{QGroundControl}
\newacronym{mavlink}{MAVLink}{Micro Air Vehicle Link}
\newacronym{xml}{XML}{eXtensible Markup Language}
\newacronym{qml}{QML}{Qt Modeling Language}
\newacronym{kml}{KML}{Keyhole Markup Language}
\newacronym{gps}{GPS}{Global Positioning System}
\newacronym{dem}{DEM}{Digital Elevation Model}

%%%%%%%%%%%%%%%%%%%%%%%%%%%Acronyms list END%%%%%%%%%%%%%%%%%%%%%%%%%%%
\begin{document}

\setcounter{secnumdepth}{3}
\setcounter{tocdepth}{3}
\hypersetup{pageanchor=false}
 \maketitle
% \includepdf[pages=-]{bambithesis-frontespizio.pdf}

	
\thispagestyle{empty}
 \setcounter{page}{2}
 \cleardoublepage


\pagenumbering{roman}
\begin{abstract}
\addcontentsline{toc}{chapter}{Abstract}{}{}
\thispagestyle{plain}
\setcounter{page}{3}

Environment representation and path planning are two major issues in any navigation systems.
This thesis discuss those issues in the specific use case regarding search and rescuing operation over agricultural area.\par
First, it is analyzed the process of image orthoretification and KML (Keyhole Markup Language) file format for representing geographical features, i.e the field boundary.
% , then it focus on Coverage Path Planning strategies.
Orthorectification is the process of correcting geometrical distortions of an image such that the scale is uniform and it can be use to obtain environment-specific information which are later encoded in KML.\par
Successively, the focus moves on Coverage Path Planning (CPP) that is the operation of finding a path covering all the points of a specific area. This task is fundamental to many robotic applications such as cleaning, painting, mine sweeping, monitoring, searching and rescue operations. The proposed approach is implemented in Robot Operating System (ROS) and it is based on approximate cellular decomposition of the environment. The coverage path is calculated using distance transform which propagates around the goal point in a wavefront. The algorithm could be adapted to optimize different path features (e.g. number of turns, path length), an important aspect especially in \acrshort{uav} application where the limited battery life requires the trajectory to be as much efficient as possible.\par
The results obtained in a simulated environment are satisfactory and even if the outcome path is not always optimal, it shows the versatility of the algorithm when applied to different field geometries.

\vskip 2em
\keywords{Georeferencing, Orthophoto, Coverage Path Planning (CPP), Approximate Cellular Decomposition, Unmanned Aerial Vehicles (UAV), Multicopter}
\end{abstract}

\thispagestyle{empty}
\setcounter{page}{4}
\cleardoublepage


\renewcommand{\abstractname}{Abstract (italiano)}
\begin{abstract}
\addcontentsline{toc}{chapter}{Abstract (italiano)}{}{}
\thispagestyle{plain}
\setcounter{page}{5}
La rappresentazione dell'area di lavoro e la pianificazione del percorso sono due aspetti importanti in qualsiasi sistema di navigazione.
Questa tesi si propone di discutere tali questioni nello specifico ambito delle operazioni di ricerca e salvataggio della fauna selvatica su suolo agricolo.\par
Inizialmente, viene analizzato il processo di ortorettificazione di immagini e presentato KML (Keyhole Markup Language) quale formato standard di rappresentazione di dati geografici.
L'ortorettificazione consiste nel correggere distorsioni geometriche dell'immagine così da rendere la scala uniforme e usarla per ottenere informazioni specifiche sul campo di lavoro da codificare successivamente in KML.\par
Proseguendo, l'attenzione si sposta sulla Pianificazione del Percorso di Copertura (CPP), ovvero su quell'operazione che consente di trovare il percorso che copra tutti i punti di un'area designata e che per tale aspetto risulta fondamentale in differnti applicazioni in robotica quali pulizia, pittura, disinnesco mine, monitoraggio, e operazioni di ricerca e salvataggio. L'approccio proposto è implementato su ROS (Robot Operating System) e si basa sulla decomposizione approssimata dell'area in celle. Il tracciato di copertura è calcolato tramite una trasformazione della distanza che si espande dal punto d'arrivo in un fronte d'onda. L'algoritmo può essere adattato per ottimizzare diverse caratteristiche della traiettoria generata (e.g. numero di curve, lunghezza del percorso), aspetto importante specialmente per applicazioni riguardanti \acrshort{uav} dove la limitata durata della batteria necessita che la traiettoria sia la più efficiente possibile. \par
I risulatati ottenuti in ambiente simulato sono soddisfacenti e, nonostante il percorso generato non sia sempre quello ottimo, diversi test svolti su campi di forma e dimensione diversi dimostrano la versatilità dell'algoritmo.

\vskip 2em
\keywords{Georeferenziazione, Ortofotografia, Pianificazione Percorso di Copertura (CPP), Decomposizione Cellulare Approssimata, Aeromobile a Pilotaggio Remoto (APR), Multicottero}
\end{abstract}



\thispagestyle{empty}
\cleardoublepage

% use pagestyle plain for content to put page numbers in the footer
\hypersetup{pageanchor=true}
\pagestyle{plain}
\tableofcontents


% Print Acronym list
\printglossary[type=\acronymtype]
\cleardoublepage

% begin content and use roman numbers for chapter,
% without including them in the section numbering (no I.1)
\pagestyle{fancy}
\pagenumbering{arabic}
\renewcommand{\thechapter}{\Roman{chapter}}
\renewcommand*\thesection{\arabic{section}}
\chapter{Introduction} % (fold)
\label{cha:introduction}
%!TEX root = bambi-thesis.tex



In this introduction it is first briefly described the motivation, objective and scope of the project this thesis has been developed. Then, the Thesis outline for every Chapter is described.


\section{Motivation} % (fold)
\label{sec:motivation}
 Deer gives birth to their offspring in spring, especially during April and May \cite{MowlingMortality}, often choosing meadows as they consider it a safe spot. This period is unfortunately the same in which meadows are cut. The results is that every year a great number of young deer fall victim of combine harvesters cutting hay. Germany counts about 100000 death every harvest season \cite{MowlingMortality}. The BambiSaver project was born with the aim to provide an autonomous, fast and user friendly device able to detect and localize living creatures in agricultural area.




Parlare un po dell'idea di base 




 Germany, as well as other country are seeking for solution to this tragic localize animal


A German wildlife rescue project is deploying small aerial drones to find young deer hiding in tall grass and protect them from being shredded by combine harvesters cutting hay in spring.{}

The proposal thesis was designed 

\subsection{Importance in agricultural}
\lipsum[7]
\index{Agriculture}
\todo{fixthis}

% section motivation (end)

\section{State of the art} % (fold)
\label{sec:state_of_the_art}

Some solutions already exist, but they needs at least two people: a pilot and an operator constantly watching at the thermal camera live stream.

\lipsum[47]

See \cite{KJ:2016}



% section state_of_the_art (end)

\section{Innovation} % (fold)
\label{sec:innovation}

\lipsum[2]

\begin{align}
  \vec F &= \vec a \times \vec b\\
  {dof}_{rot} &= \sum_{i=1}^n (i-1) = n\, \frac{n-1}{2}
\end{align}

\lipsum[7-13]
\todo{test}
\lipsum[14-20]

% section innovation (end)
% chapter introduction (end)
\chapter{Georeferencing the mission's environment} % (fold)
\label{cha:georeferencing_the_mission_s_environment}
%!TEX root = bambi-thesis.tex
In this chapter it is explained in details how the device obtain the information of the field boundary in form of a list of geographic coordinates.
This important task will define the mission range of action which is used as input to the Coverage Path Planning module. It is thus important that the gathered information are precise and consistent or it will impact the successive steps of the mission.\\
The process could be divided in two tasks:
\begin{itemize}
	\item Obtain a georeferenced photo displaying the entire meadow.
	\item Detect and trace the field boundary over the georeferenced image.
	\item Store the boundary as a set of geographic coordinate points.
\end{itemize}

\section{Georeferencing a Photo} % (fold)
\label{sec:georeferenced_photo}
Before an aerial image can be used to support a site-specific application it is essential to perform the geometric corrections and geocoding. This is commonly called \textit{georeferencing} which enables the assignment of ground coordinates to the different features in the datasets. If the map projection (and map projection parameters) of the ground coordinates are known, equivalent geographic coordinates can be produced which enables positioning the features of the coverage into a World context. \cite{georefPractice}.
For this specific use case the final result is an image where every point it is associated to a geographic coordinate.

\subsection{Theory Background} % (fold)
\label{sub:theory_background}

% subsection theory_background (end)

%DA USARE QUANDO PARLO DI GOOLGE MAPS E PER LA PRECISIONE ECC... e magari cosi' introduco l'idea dell'orthophoto per migliorare la precisione
Spatial datasets, like any type of data, are prone to errors. Thus, three fundamental concepts have to be kept in mind – precision, bias and accuracy. Precision refers to the dispersion of positional random errors and it is usually expressed by a standard deviation. Bias, on the other hand, is associated with systematic errors and is usually measured by an average error that ideally should equal zero. Accuracy depends on both precision and bias and defines how close features on the map are from their true positions on the ground [6]. So, despite being frequently confused concepts, high precision does not necessarily mean high accuracy. But both depend greatly on the map scale. All maps have inherent positional errors, which depend on the methods used in the construction of the map. The scale is the ratio between a distance on the map and the corresponding distance on the ground. The maximum acceptable positional error (established by cartographic standards) is determined by the map scale.
% section georeferenced_photo (end)

Fist of all a georeferenced photo containing the whole field is 

% chapter georeferencing_the_mission_s_environment (end)
\chapter{Coverage Path Planning} % (fold)
\label{cha:coverage_path_planning}
%!TEX root = bambi-thesis.tex
This chapter begins with a theory digression about the topic of Coverage Path Planning. Then the algorithm chosen for the application in this thesis is explained in details and finally the ROS implementation of the algorithm is presented and discussed.
\section{Theory Background} % (fold)
\label{sec:theory_background}
\textit{Coverage path planning} (CPP) is the problem of finding a trajectory for a mobile robot such that a target area is completely swept by the sensor footprint. In the following section it is first presented the conventional CPP algorithms in use for mobile robots. Later, the focus moves more specifically over aerial application showing some previous work regarding this topic.

\subsection{Coverage Path Planning for Mobile Robots} % (fold)
\label{sub:coverage_path_planning_for_mobile_robots}
The problem of finding an optimal coverage path, even for a simple polygon, is classified as NP-hard \footnote{NP-hard problems are problems for which there is no known polynomial algorithm, so that the time to find a solution grows exponentially with problem size. Although it has not been definitively proven that, there is no polynomial algorithm for solving NP-hard problems, many eminent mathematicians have tried and failed.} \cite{ARKIN200025}. Hence, existing approaches try to find an approximate solution which fits at best the specific application requirements. For 2D coverage, some methods decompose the target area into simpler polygons and for each compute the coverage path. Other methods use a grid-based representation which leads to an approximate coverage. Finally, closed-loop control methods avoid the needs of an a priori representation of the target region.

\subsubsection{Exact Cellular Decomposition} % (fold)
\label{ssub:exact_cellular_decomposition}
One of the main approach in area coverage path planning is based on the divide-and-conquer strategy.
In this method the target area is decomposed in simple regions called cells. Since all cells have a simple structure, each can be covered with simple motions such as back-and-forth motion as in \autoref{fig:lawnmower-pattern}. This kind of motion is called \textit{Lawnmower pattern}. Once the robot visits all cell, coverage is achieved.\\
\textit{Trapezoidal decomposition} is the most popular cell decomposition. This decomposition relies heavily on the polygonal representation of the planar configuration space \cite{book:655068}. Cells are in fact obtained simply by sweeping a vertical line through the 2D plane and, upon reaching each vertex of the environment polygon, the required edges are added to create trapezoids (see \autoref{fig:trapezoidal-decomposition}). Two cells sharing a common boundary are defined as adjacent and accordingly an adjacent graph is produced. At this point the strategy consist in finding the exhaustive walk which visits all cells and minimizes the cost of traveling between them.
An important factor in finding an efficient path is the choice of the swiping direction line when decomposing the target area as analyze by Oksane \cite{TrapezoidalDecompCPP}. He performed a local optimization to find the direction for the sweeping line which minimize trajectory length and the number of turnings.
\begin{figure}[ht]
    \centering
    \includegraphics[width=0.4\textwidth]{figures/C3/LawnmowerPattern.png}
    \caption{Lawnmower pattern}
    \label{fig:lawnmower-pattern}
\end{figure}

\begin{figure}[ht]
    \centering
    \includegraphics[width=0.5\textwidth]{figures/C3/TrapezoidalDecomposition.png}
    \caption{Trapezoidal decomposition and adjacent graph \cite{book:655068}}
    \label{fig:trapezoidal-decomposition}
\end{figure}
% subsubsection exact_cellular_decomposition (end)



% subsection coverage_path_planning_for_mobile_robots (end)
\subsection{Aerial Coverage using UAVs} % (fold)
\label{sub:aerial_coverage_using_uavs}


% The general CPP methods introduced in Chapter 2.2.1 have been applied directly to aerial coverage often in an offline setting. In these works, the target area is specified as a polygon or grid with GPS coordinates of the polygon nodes or grid center. It is usually assumed that the UAV is flying at an altitude which is safe and hence there is no risk of colliding with an obstacle. However sometimes some parts of the environment are treated as no-flying zones which should be avoided by the UAV. Therefore in CPP for UAVs, these subregions are dealt with as obstacles. Another type of region that might be present in aerial coverage is uninteresting regions. Flying in these regions is allowed but does not contribute to the coverage goal i.e. covering them is of no value. For example in a outdoor crop mapping task with a quadrotor, covering the lake is unnecessary but flying over it (for example to reach the other side of the lake) is allowed.\\
% As mentioned before, due to the very short flight time of UAVs, generating efficient coverage paths is useful. To reach efficiency, existing methods try to avoid unnecessary coverage, i.e. they try to minimize the overlap in the sensor footprint along the produced trajectory. This goal will consequently reduce the path length. Another element considered by some existing methods is to minimize the number of turns in the flight trajectory. Reducing the number of turnings will consequently produce trajectories that consist of long straight stripes, as is the case in Lawnmower pattern. Therefore, when a robot follows the trajectory it can maintain a constant velocity in a large part of the coverage path and it only accelerates or de-accelerates when it is turning. The overall outcome is less power consumption. Such efficient coverage trajectories can be planned offline when the environment is known a priori. Some of these methods, aimed for aerial coverage, are presented in the following section which includes planning for both single and multi-robot systems. However, when no prior knowledge about the target area is available, the planning has to be done online based on the real-time sensory data.\\
% In the following, we present the existing methods of coverage trajectories planning for unmanned aerial vehicles. We partition the approaches into two main groups: a) methods that precompute the trajectory based on a priori knowledge of the environment and b) methods that adaptively re-plan the trajectory based on on-line sensory data.
 % subsection aerial_coverage_using_uavs (end)

\subsection{Flight Altitude} % (fold)
\label{sub:flight_altitude}
How to compute the required flight altitude (resolution + sensor footprint + required photo overlap). Some calculation will be listed.
% subsection flight_altitude (end


% section theory_background (end)

\section{Proposal Solution} % (fold)
\label{sec:proposal_solution}

Abbiamo scelto il wavefront perche' oltre ad essere abbastanza efficente computazionalmente permette di tenere conto di diverse cost function (diverse priorita' come ad esempio l'altitudine, minor number of turns, ecc...)]
% section proposal_solution (end)


\section{Implementation in ROS} % (fold)
\label{sec:implementation_in_ros}


% section cpp_algorithms (end)
% chapter coverage_path_planning (end)
\chapter{Simulation Results} % (fold)
\label{cha:simulation_results}
%!TEX root = bambi-thesis.tex
% \section{Simulation Results} % (fold)
% \label{sec:simulation_results}
The PX4 flight stack firmware offers a great \acrfull{sitl} simulation environment which was fundamental in develop the software. In fact, it provides a real-time, safe and convenient way to test the software implementation without all the risks related to a real flight. \\

\section{Simulation Environment} % (fold)
\label{sec:simulation_environment}
The simulation environment consist of:
\begin{itemize}
 	\item \textbf{PX4 \acrshort{sitl}}: The PX4 simulated hardware which reacts to the simulated given input exactly as it would react in the reality and issues the output as a percentage of the total thrust that every rotor has to provide.
 	\item \textbf{Gazebo}: The dynamics simulator that is used by the SITL. This software reads the PX4 output and, by elaborating the modeled dynamics, provides the simulated input to the PX4. This allows for a quite accurate simulation of the real model behavior during flight.
 	\item \textbf{\acrshort{ros}}: the robotics middleware over which the Bambi Project software (in the form of package) runs.
 	\item \textbf{\acrshort{qgc}}: the ground control station used to send the mission start/stop (see appendix \ref{appendix:mavlink}) command and to monitor all the \acrshort{uav} parameters during flight.
 \end{itemize}
 The different parts of the system (\autoref{fig:SITL-architecture}) are connected via UDP, and can be run on either the same computer or another computer on the same network. The communication protocol is MAVLink (see appendix \ref{appendix:mavlink}). \\
 The following simulations has been carried out on a single PC running Ubuntu 16.04LST.
 \begin{figure}[ht]
    \centering
    \includegraphics[width=.7\textwidth]{figures/C4/Px4_sitl_overview}
    \caption{SITL with Gazebo architectural scheme}
    \label{fig:SITL-architecture}
\end{figure}

Gazebo offers the possibility to load a satellite map through the \textsf{StaticMap} plugin\footnote{documentation at \url{http://gazebosim.org/tutorials?tut=static_map_plugin&cat=build_world}}. In this way the simulation environment (\autoref{fig:gazebo-Map}) represents \todo{depicts????} quite accurately the real world scenario.
The vehicle used in simulation environment is the 3DR Iris (\autoref{fig:gazebo-iris}), because its model was already created by PX4 team along with different sensors like the 2D 360\degree\ lidar that was used to test object avoidance.
\begin{figure}[ht]
  \centering
  \includegraphics[width=.7\linewidth]{figures/C4/simulation/gazebo-bambi-world.jpg}
  \caption{Gazebo satellite ground plane}
  \label{fig:gazebo-Map}
\end{figure}
\begin{figure}[ht]
  \centering
  \includegraphics[width=.5\linewidth]{figures/C4/simulation/iris-model1.jpg}
  \caption{Gazebo 3DR iris model}
  \label{fig:gazebo-iris}
\end{figure}
% section simulation_Environment (end)

\section{Starting Bambi Mission} % (fold)
\label{sec:starting_bambi_mission}
Mission start and stop command are issue using the \acrshort{qgc} custom command widget displayed in \autoref{fig:qgc-widget}. This widget was written in \acrfull{qml} an user interface markup language providing an easy way to write simple GUIs \cite{QML}.\\
The button \textsf{BAMBI MISSION START} automatically send the mission start message while \textsf{BAMBI MISSION STOP} send the mission abort message which according to the current mission state causes the \acrshort{uav} to return to the home position or to immediately land at the current position.
\begin{figure}[ht]
  \centering
  \includegraphics[width=.9\linewidth]{figures/C4/simulation/qgc-widget.png}
  \caption{QGC Bambi's custom command widget}
  \label{fig:qgc-widget}
\end{figure}
% subsection starting_bambi_mission (end)

\section{Testing CPP algorithm} % (fold)
\label{sec:testing_cpp_algorithm}
In order to test the \acrlong{cpp} algorithm it would be necessary to run through the whole mission steps and until the mission controller issue the \textsf{FieldCoverageInfo} message triggering the \acrshort{cpp} node. Therefore to save the time required by the drone to perform the first mission tasks it is convenient to publish a "fake" message directly over the \textsf{/bambi/mission\_controller/trigger\_path\_generation} topic. This is performed executing a simple bash script calling the \textsf{rostopic pub} command.\par
The algorithm was tested using different sample fields. The results obtained for three of them are displayed in \autoref{fig:Coverage-pathGE}. The cellular decomposition consist of $10 \times 10$ cells. \autoref{tbl:fields-area-length} reports the path length compared to the field area which as expected is greater for larger fields.

\begin{table}[ht]
\centering
\begin{tabular}{|l|l|l|}
\hline
\multicolumn{1}{|c|}{\textbf{Field}} & \multicolumn{1}{c|}{\textbf{Area}} & \multicolumn{1}{c|}{\textbf{Path Length}} \\ \hline
1                                    & $15444\, m^2=1.5444\, ha$          & $2.17\, Km$                               \\ \hline
2                                    & $50665\, m^2=5.0665\, ha$          & $5.56\, Km$                               \\ \hline
3                                    & $41586\, m^2=4.1586\, ha$          & $4.66\, Km$                               \\ \hline
\end{tabular}
 \caption{Comparison between field extension and coverage path length}
 \label{tbl:fields-area-length}
\end{table}
\begin{figure}[ht]
	\centering
	\begin{subfigure}{.49\textwidth}
	  \centering
	  \includegraphics[width=.95\linewidth]{figures/WaldpeterField/Waldpeter-CoverageGE.jpg}
	  \caption{Field 1}
	  \label{sfig:F1-CPP-path}
	\end{subfigure}
	\begin{subfigure}{.5\textwidth}
	  \centering
	  \includegraphics[width=.95\linewidth]{figures/Field2/Field2-CoverageGE.jpg}
	  \caption{Field 2}
	  \label{sfig:F2-CPP-path}
	\end{subfigure}
	\begin{subfigure}{.49\textwidth}
	  \centering
	  \includegraphics[width=.95\linewidth]{figures/Field3/Field3-CoverageGE.jpg}
	  \caption{Field 3}
	  \label{sfig:F3-CPP-path}
	\end{subfigure}
	\caption{CPP output path applied to different shaped fields}
    \label{fig:Coverage-pathGE}
\end{figure}
% subsection testing_cpp_algorithm (end)

\subsection{Different Starting and Goal Positions} % (fold)
\label{sub:different_starting_and_goal_positions}
During the design of {}the \acrshort{cpp} node several starting and goal position were tested. This helped to understand how the algorithm behaves under different condition and to implement a smart selection of the starting cell. \autoref{fig:coverage-start-goal-pos} shows the outcome path under three different situations.\par
When the starting point is in the middle of the field, as it is in \autoref{sfig:F3-modified-start-pos}, the coverage trajectory starts as a straight line toward the furthest point from the goal position, thought it is the path having the steepest ascending gradient. The cells visited in this transition are marked as visited and therefore avoided in the successive steps. This leads to have such kind of two halves partitioning in the trajectory.\par
The situation radically change if, at the center of the workplace, is placed the goal point. The wave-front propagate all around this position and the generated path assume the spiral like shape in \autoref{sfig:F3-modified-start-goal-pos}. As in the previous situation the path present numerous changes of direction.\par
Finally in \autoref{sfig:F3-regular-start-pos} the starting cell is chosen looking for the farthest most isolated cell with the method presented in \autoref{sub:choosing_starting_position}. The goal position is instead at the extremity of the field and it coincides with the home position, thus the point where the \acrshort{uav} were placed when the mission started. This is reasonable as, in a real situation, where the grass is tall, the mission is suppose to start from the field's border. The resulting path is definitely the best of the three showing a lawnmower like behavior. The prevalence of straight lines limits speed variations, consequently acceleration and power consumption are reduced.
\begin{figure}[ht]
	\centering
	\begin{subfigure}{0.49\textwidth}
	  \centering
	  \includegraphics[width=0.95\linewidth]{figures/Field3/Field3-CoverageGE-BSpline-noBorder.jpg}
	  \caption{}
	  \label{sfig:F3-regular-start-pos}
	\end{subfigure}
	\begin{subfigure}{0.49\textwidth}
	  \centering
	  \includegraphics[width=0.95\linewidth]{figures/Field3/Field3-CoverageDifferentStartingPosition.jpg}
	  \caption{}
	  \label{sfig:F3-modified-start-pos}
	\end{subfigure}
	\begin{subfigure}{0.49\textwidth}
	  \centering
	  \includegraphics[width=1\linewidth]{figures/Field3/Field3-CoverageDifferentStartingGoalPosition.jpg}
	  \caption{}
	  \label{sfig:F3-modified-start-goal-pos}
	\end{subfigure}
	\caption{Coverage Path generated with Starting and Goal position}
    \label{fig:coverage-start-goal-pos}
\end{figure}
% subsubsection different_starting_and_goal_positions (end)
\subsection{Errors Analysis} % (fold)
\label{sub:errors_analysis}
Looking at the three coverage paths in \autoref{fig:Coverage-pathGE} one can notice that in some section at the border, the trajectory exceed the boundary (represented as black line). This phenomenon is a drawback of the approximate cellular decomposition (see \autoref{sub:approximate_cellular_decomposition}. The algorithm, in fact, marks a cell as part of the field when the intersection between the square polygon of the cell and the field polygon
exist. This cause the cells locate at the edge of the workplace to be marked as field even if just a small portion of them are actually inside the field. This approximation leads to imperfections in the path as shown in \autoref{fig:CPP-imperfections}. The larger are the cells, the greater becomes the effect because the approximation gets worse. For instance using $10 \times 10\, m$ cells the trajectory goes outside the border of a distance as high as $9\, m$.\par
Possible \textit{solutions} could be:
 \begin{enumerate*}
	\item Using a grid with smaller cell,thus increasing the sensor overlap;
	\item Considering a threshold on the intersecting area under which the cell it is no considered as part of the field. This could lead to a not complete coverage of the target area.
	\item In the algorithm handle external cells as particular case and using as waypoint not the center but a point inside the workplace.
\end{enumerate*}

\begin{figure}[ht]
	\centering
	\begin{subfigure}{.49\textwidth}
	  \centering
	  \includegraphics[width=.9\linewidth]{figures/Field3/Field3-imperfection.png}
	  \caption{}
	  \label{sfig:CPP-imperfection1}
	\end{subfigure}
	\begin{subfigure}{.49\textwidth}
	  \centering
	  \includegraphics[width=.95\linewidth]{figures/Field3/Field3-imperfection2.png}
	  \caption{}
	  \label{sfig:CPP-imperfection2}
	\end{subfigure}
	\caption{Errors due to Approximate Cellular Decomposition}
    \label{fig:CPP-imperfections}
\end{figure}
% subsubsection errors_analysis (end)
\section{Final Considerations} % (fold)
\label{sec:considerations}
Overall, the generated path (assuming a smart selection of start and goal point) is satisfactory for the desired purpose even if, in some section, there are turns which would be nice to avoid. This is accentuated especially when the workplace has a complex shape (\autoref{sfig:F2-CPP-path}). This shortcoming is something expected, in fact, at the actual development stage only the distance from the goal point is used as cost function in the wave-front propagation algorithm, therefore no optimization for reducing turns has been implemented (see \autoref{sub:wave_front_propagation_algorithm}). The distance transformation used, in fact, leads to a spiral like behavior near the goal cell (discussed in \autoref{ssub:grid_based_methods}), thus it is convenient to set the goal position (home position in the mission) at an extremity of the field. \par
The issue discussed in \autoref{sub:errors_analysis} could cause real problem if at the border of the field there are obstacles risking to make the \acrshort{uav} crash. This, at the actual state of development, is partially addressed relying on the \textit{object avoidance} system that has been implemented inside \textit{PX4 Firmware} which, using potential field, try to push the drone away from surrounding obstacles.\par
Finally, considering that the average flight time of a mid-range drone is between $15$ to $20$ minutes and assuming a flight mean velocity of about $4.5\, m/s$ during the mission duration, the distance it can travel is in the range of $4-5.4\, Km$. This distance is enough to cover a $4\, Ha$ field (\autoref{tbl:fields-area-length}).
% subsubsection considerations (end)


% \textit{Qui mostrero' l'output dell'algoritmo rappresentato come tracciato su google earth. verranno messi a confronto diverse scelte di starting e goal point e discutero' delle performance ottenute con relativi problemi da risolvere}

% \textit{\textbf{Mostrare importanza della scelta della starting position + bezier vs spline interpolation}}
% section simulation_results (end)


















 % Da mettere nel corpo della tesi NON INTRO

% \section{Hardware Setup} % (fold)
% \label{sec:hardware_setup}
% The quadcopter (UAV) is equipped with:
% \begin{itemize}
% 	\item Pixhawk flashed with PX4 flight stack (see appendix \ref{appendix:pixhawk_flight_controller})
% 	\item NEO-M8n (GPS){}
% 	\item 3DR telemetry radio 433Hz (serial link between the UAV and the ground station)
% 	\item LidarLite V3 (altitude distance sensor) \cite{grm:lidarlite}
% 	\item Raspberry Pi 3b (Onboard computer running ROS over Ubuntu OS)
% \end{itemize} 
%  The ground station is composed by a common laptop running QGroundControl \todo{appendix or small description and features of QGC} application. The communication with the UAV uses the MAVLink protocol \cite{Mavlink} through an USB 433Hz telemetry radio .
% % section hardware_setup (end)

% \section{Software} % (fold)
% \label{sec:software}

% \todo{ROS, PX4 why we choose them}
% \todo{ROS design graph and explanation of each node}
% \todo{go deeply in ortho photo a}
% % section software (end)
% section  (end)
% chapter simulation_results (end)

% chapter simulation_results (end)
\chapter{Conclusion and Future Work} % (fold)
\label{cha:conclusion}
\input{c5-Conclusion}
% chapter conclusion (end)


\bibliography{bambi-thesis}
\bibliographystyle{IEEEtran}


\begin{appendices}

%\chapter{Supplementary Information} % (fold)
%\label{cha:supplementary_information}
%!TEX root = bambi-thesis.tex
\chapter{ROS} % (fold)
\label{appendix:ros}
\textit{"The Robot Operating System (ROS) is an opensource software framework supporting the development of complex, but modular systems in a distributed computing environment. While the core components of ROS are highly generic, the primary focus of ROS and its ecosystem is set to the development and research of robots. The performance critical parts of the framework are written in C++, but applications operating on top of the framework may currently be written in C++, Python or Lisp."}\cite{7795766} \par

\section{Concepts} % (fold)
\label{sec:concepts}
\begin{description}
    \item[Master] It provides name registration and lookup to the rest of the Computation Graph. Without the Master, nodes would not be able to find each other, exchange messages, or invoke services.
    \item[Nodes] are processes that perform computation. ROS is designed to be \textit{modular} at a fine-grained scale; a robot control system usually comprises many nodes. For example, one node controls a laser range-finder, one node controls the wheel motors, one node performs localization, one node performs path planning, one Node provides a graphical view of the system, and so on. A ROS node is written with the use of a ROS client library, such as roscpp or rospy
    \item[Parameter Server] It allows data to be stored by key in a central location. It is currently part of the Master.
    \item[Messages] Nodes communicate with each other by passing messages. A message is simply a data structure, comprising typed fields. Standard primitive types (integer, floating point, boolean, etc.) are supported, as are arrays of primitive types. Messages can include arbitrarily nested structures and arrays (much like C structs).
    \item[Topics] Messages are routed via a transport system with publish / subscribe semantics. A node sends out a message by publishing it to a given topic. The topic is a name that is used to identify the content of the message. A node that is interested in a certain kind of data will subscribe to the appropriate topic. There may be multiple concurrent publishers and subscribers for a single topic, and a single node may publish and/or subscribe to multiple topics. In general, publishers and subscribers are not aware of each others’ existence. The idea is to decouple the production of information from its consumption. Logically, one can think of a topic as a strongly typed message bus. Each bus has a name, and anyone can connect to the bus to send or receive messages as long as they are the right type.
    \item[Services] The publish / subscribe model is a very flexible communication paradigm, but its many-to-many, one-way transport is not appropriate for request / reply interactions, which are often required in a distributed system. Request / reply is done via services, which are defined by a pair of message structures: one for the request and one for the reply. A providing node offers a service under a name and a client uses the service by sending the request message and awaiting the reply. ROS client libraries generally present this interaction to the programmer as if it were a remote procedure call.
    \item[Bags]A format for saving and playing back ROS message data. Bags are an important mechanism for storing data, such as sensor data, that can be difficult to collect but is necessary for developing and testing algorithms.
\end{description}
\begin{figure}[ht]
    \centering
    \includegraphics[width=1\textwidth]{figures/A1/ROS-master-node-topic.png}
    \caption{ROS Nodes communication}
    \label{fig:ROS-architecture}
\end{figure}
% section concepts (end)
% chapter ros (end)
%!TEX root = bambi-thesis.tex

\chapter{Pixhawk Autopilot} % (fold)
\label{appendix:pixhawk_flight_controller}
\section{Hardware} % (fold)
\label{sub:hardware}
\textit{“Pixhawk is an independent, open-hardware project aiming at providing high-end autopilot hardware to the academic, hobby and industrial communities at low costs and high availability. It provides hardware for the Linux Foundation DroneCode project. It originated from the PIXHAWK Project of the Computer Vision and Geometry Lab of ETH Zurich (Swiss Federal Institute of Technology) and Autonomous Systems Lab as well from a number of excellent individuals.”} \cite{Pixhawk}
\begin{figure}[ht]
    \centering
    \includegraphics[width=.6\textwidth]{figures/A2/pixhawk.jpg}
    \caption{}
    \label{fig:Pixhawk Flight Controller}
\end{figure}

The Pixhawk hardware weights 38g and it is provided with a 32-bit ARM Cortex M4 core with FPU with 256 KB of RAM and a 32-bit fail-safe co-processor; it is also equipped with a compass, a barometer, an accelerometer and a gyro sensor. 
% section hardware (end)

\section{Softwere} % (fold)
\label{sub:softwere}
Modern, sophisticated flight controllers share a commonality in architecture. We can divide their functionality into three distinct layers, illustrated in \autoref{fig:FCU-architecture}.
\begin{figure}[ht]
    \centering
    \includegraphics[width=.5\textwidth]{figures/A2/flightControllerArchitecture.png}
    \caption{The architecture of a modern flight controller}
    \label{fig:FCU-architecture}
\end{figure}
\paragraph{Layer 1: Real Time Operating System\\} % (fold)
\label{par:layer_1_real_time_operating_system}
The real time operating system is the back bone of the flight firmware, providing basic hardware abstraction and concurrency. Real time systems are critical for flight control performance and safety, as they guarantee that flight control tasks will be completed in a certain amount of time, and are essential for the safety and time-critical performance of UAVs. Luci uses a real time operating system called NuttX, which is highly expansive and configurable.
% paragraph layer_1_real_time_operating_system (end)

\paragraph{Layer 2: Middleware\\} % (fold)
\label{par:layer_2_middleware}
The middleware is a collection of tools, drivers, and libraries that relate to flight control. It contains device drivers that handle sensors and other peripherals. It also contains flight control libraries such as RC protocols, math utilities, and control filters.
% paragraph layer_2_middleware (end)

\paragraph{Layer 3: Flight Control\\} % (fold)
\label{par:layer_3_flight_control}
The flight control layer is the brains of the operation; this layer contains all of the command and control routines. Things like state estimation, flight control, system calibration, telemetry, motor control, and other flight control aspects reside in this layer.
% paragraph layer_3_flight_control (end)

\subsection{PX4 Stacks} % (fold)
\label{par:avaiable_flight_controller}

% subsection avaiable_flight_controller_ (end)
The Pixhawk platform can be flashed with two very popular flight stack: ArduPilot and PX4.
In this thesis it has been adopted the PX4 software because:
\begin{itemize}
	\item Its software-in-the-loop (SITL) simulation is much more developed and matured.
 	\item Supports a much larger number of peripherals, including more IMU sensors, lidar, range finders, status indicators, optical flow, and motion capture units. PX4 supports the most advanced sensing peripherals for drones.
 	\item Contains advanced command and control functionality, including things like terrain estimation, and indoor flight correction.
 	\item More ubiquitous and built with advanced drone applications in mind. It can be compiled for POSIX (Linux) systems, and it can also integrate with ROS to run flight applications in a hybrid system, with some running on an underlying real-time OS, and others running on Linux using ROS to communicate.
 \end{itemize} 
 The choice was mainly driven by the more developed SITL environment and framework provided by the PX4 community and for the more advanced integration with ROS rather then a matter of performance or features, where ArduPilot stack prove to be good as well.
 The diagram in \autoref{fig:PX4-architecture} provides a detailed overview of the building blocks of PX4. The top part of the diagram contains middleware blocks, while the lower section shows the components of the flight stack.
 \begin{figure}[ht]
    \centering
    \includegraphics[width=.6\textwidth]{figures/A2/PX4_Architecture.pdf}
    \caption{PX4 Architecture}
    \label{fig:PX4-architecture}
\end{figure}
% section softwere (end)

% section pixhawk_flight_controller (end)
\input{a3-mavlink}
%!TEX root = bambi-thesis.tex

\chapter{Mavros} % (fold)
\label{appendix:mavros}
Mavros package implements a MAVLink extendable communication node for ROS with UDP proxy for Ground Control Station that includes the following features:
\begin{itemize}
	\item Communication with autopilot via serial port
	\item UDP proxy for Ground Control Station
	\item Plugin system for ROS-MAVLink translation
	\item Parameter manipulation tool
	\item Waypoint manipulation tool'
\end{itemize}

 \begin{figure}[ht]
    \centering
    \includegraphics[width=.7\textwidth]{figures/A4/diagram.png}
    \caption{Mavros as communication gateway}
    \label{fig:mavros-diagram}
\end{figure}
Through mavros it is possible to communicate with the flight controller (\autoref{fig:mavros-diagram}) simply publishing over the topic subscribed by mavros or making a call to the services it provides.
The package is made up of various plugins, each handling different part \todo{trovare sinonimo di part} of the FCU.
Every plugin can be load and configure separately when starting mavros through \textit{launch} file. Inside the package there are already some sample launch files specifically created to configure the communication with PX4 or APM flight stack.
\todo{listare i plugin usati con PX4.launch???}

% chapter mavros (end)
\end{appendices}

\pagebreak
\listoffigures
\pagebreak
\listoftables

% \lstlistoflistings

% chapter supplementary_information (end)




\end{document}
